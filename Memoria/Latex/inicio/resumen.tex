
Los procesos de intercambio de gases entre la tierra y la atmósfera determinan una parte importante del clima, por lo tanto es fundamental poder obtener datos fiables sobre su comportamiento y entender los mecanimos que lo determinan. Entre los diferentes factores que afectan al comportamiento de los gases se pueden encontrar, la temperatura, la humedad, la presión y el viento. Las concentraciones y flujos de gases fluctuan por la atmósfera debido al viento, es por esto que es importante realizar un estudio de este parámetro.

En este trabajo de final de grado se detalla el proceso de diseño, desarrollo y construcción de un dron como plataforma móvil para la instalación de un sensor de viento. Además, este sistema es capaz de  acoplar otros sistemas de sensorización miniaturizados, de esta manera, el dron puede incluir sensores de viento, sensores de medición de gases, de temperatura, de presión y de humedad. La plataforma móvil construida para esta unificación es un cuadrotor, dron con 4 motores. Se ha realizado un estudio técnico de cada componente de forma individual para obtener un sistema optimizado en términos de capacidad de carga útil y tiempo de vuelo. Además, ha sido necesario diseñar y producir una serie de piezas con un programa de diseño asistido por ordenador 3D y una impresora 3D para así acoplar los sistemas de medición necesarios y convertir finalmente el dron en un sensor. Por otro lado para la gestión y obtención de los datos a de los sensores instalados, se ha tenido que diseñar y codificar un software específico en python. También se ha desarrollado un algoritmo que es capaz de obtener y analizar los parámetros de los motores convirtiendo de esta forma el dron en un sensor. Finalmente, el sensor de viento diseñado se ha implementado en base a una unidad de medición inercial, registrando inclinación, aceleración y orientación del sistema donde se ha instalado.

Después de construir e implementar los diferentes sistemas de administración y establecer la recopilación de datos, el sistema se probó en una prueba de campo y los resultados obtenidos muestran un buen funcionamiento. Por medio de la representación de los datos tomados, se observa y procesa la dirección y la velocidad del viento.