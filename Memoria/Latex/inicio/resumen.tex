
Los procesos de intercambio de gases entre la tierra y la atmósfera determinan una parte importante del clima. Por lo tanto es fundamental poder obtener datos fiables sobre su comportamiento y entender los mecanimos que lo determinan. Entre los diferentes factores que afectan al comportamiento de los gases se pueden encontrar, la temperatura, la humedad, la presión e incluso el viento. Los gases fluctuan por la atmósfera debido al viento, es por esto que es importante realizar un estudio de este parámetro.

Este trabajo de final de grado ha realizado una construcción de una dron como plataforma móvil sobre la que instalar un sensor de viento y poder acoplar otros sistemas de sensorización miniaturizados; dentro de estos sistemas de medición se pueden incluir, sensores de viento, sensores de medición de gases, de temperatura, de presión y de humedad. La plataforma móvil construida para esta unificación es un cuadrotor, dron con 4 motores, para ello se ha realizado un estudio técnico de cada componente de forma individual y posteriormente ha sido necesario diseñar una serie de piezas con un programa de diseño asistido por ordenador 3D para así acoplar los sistemas de medición necesarios y convertir finalmente el dron en un sensor. Por otro lado para la gestión y obtención de los datos a partir de los sensores instalados, se ha tenido que diseñar y codificar un software específico para dichos sensores, se ha programado en python. Finalmente, el sensor de viento diseñado se ha basado en una unidad de medición intercial, registrando inclinación, aceleración y orientación del sistema donde se ha instalado.

Una vez construido el dron e instalado los diferemtes sistemas de medición se ha efectuado la toma de datos, obteniendo unos resultados acordes con lo esperado y mostrando por tanto su buen funcionamiento; mediante la representación de los datos tomados se observa la dirección y velocidad que toma el viento.
