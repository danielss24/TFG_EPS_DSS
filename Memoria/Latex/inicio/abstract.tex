
The gas exchange processes between the earth and the atmosphere determine an important part of the climate. Therefore, it is essential to obtain reliable data on their behavior in order to understand the mechanisms that determine it. Among the different factors that affect the behavior of gases are temperature, humidity, pressure, and  wind. Gas concentrations and fluxes vary in the atmosphere due to the wind, which is why it is important to determine this parameter.

This final degree project deals with the design, development and construction of a drone as a mobile platform for the installation of a wind sensor. Additionally the system is able to couple other miniaturized, digitized sensors. This way the drone system can include wind sensors, gas sensors, temperature, pressure and humidity sensors. The mobile platform for this assembling is a quadcopter, i.e. a drone with 4 motors. A technical study of each drone component has been carried to obtain an optimized system in terms of payload and flight time. On top of that it has been necessary to design and produce a series of pieces using a 3D computer assistance design program and a 3D printer. In order to obtain and to manage the data of the various sensors, it has been necessary to design and codify a specific software in python. Furthermore, an algorithm has been developed to read-out and analyze the motor´s operational parameters to convert the drone itself into a wind sensor. Finally, a designed wind sensor has been implemented based on an inertial measurement unit, registering inclination, acceleration and orientation of the system where it is installed.

After building and implementing the different management systems and establishing data collection, the system has been tested in a field trial and the obtained results show good performance. By means of the representation of the taken data the direction and the speed that the wind can be is observed and processed further.