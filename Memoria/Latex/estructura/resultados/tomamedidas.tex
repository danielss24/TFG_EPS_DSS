
Se han realizado las tomas de medidas de formas distintas: la primera como una toma única de datos y la segunda como la toma de conjunto de puntos como el descrito en la figura \ref{FIG:PLANOGIMNASIO}.

En ambas tomas de medidas se ha alterado la altura del dron para poder realizar una comparación del viento, a una altura de 0,5 metros (m) del suelo y a 2,5 metros aproximadamente. Cabe indicar que para la toma del conjunto de datos, se ha simulado una matriz en el gimnasio y modificando la posicion del dron para cada una de las tomas de datos, cambiando tanto su posicion en el eje X, Y como Z.

Para las tomas de medidas se ha simulado una corriente de aire con un ventilador; no obstante al ser una circulación de aire tan leve se tuvo que incorporar en la generación del aire, un movimiento mecánico manual de una colchoneta a forma de abanico.

Una vez preparado el escenario, con y sin viento, se procedió a la colocación del dron y a la toma de datos. En una primera prueba se realizaron tomas de datos individuales a varias alturas para familiarizarse con el comportamiento del dron con y sin viento. En las siguientes pruebas se realizaron algunas tomas individuales y un total de 2 pruebas de matrices, con un conjunto de 8 puntos, 4 en un plano y 4 en otro plano. El primer plano se definió a una altura aproximada de 0'50 metros (m) del suelo, mientras que el segundo se situó a una altura aproximada de 2'5 m de altura. 

La forma de tomar los datos con el sensor de viento y el dron, fue iniciada y terminada en vuelo.