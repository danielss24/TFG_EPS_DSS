
Para realizar el estudio de los gases que componen la atmósfera hay que tener en cuenta su comportamiento como por ejemplo, de dónde vienen y hacia dónde van. Los gases se desplazan por dos causas distintas: por difusión y por conducción, el método de difusión consiste en el desplazamiento de los gases, generalmente, en un entorno cerrado mediante el equilibrio de sus presiones parciales. Por otro lado la conducción de gases consiste en el desplazamiento de un gas por efecto de otro gas de presión menor, como por ejemplo el viento. Para medir el viento se utilizan sistemas como la veleta potenciométrica\cite{Mur2012} para medir la velocidad y dirección del viento, pudiendo diferenciar diferentes métodos de medición de velocidad de viento como: estándar o de copas, filamento caliente, empuje, compresióno tubo de Pitot, multisondas, láser, esférico y ultrasónico, descritos en el capítulo \ref{CAP:ESTADODELARTE}. No obstante, con los avances tecnológicos desarrollados en los últimos años, se pueden utilizar sensores más modernos  como las IMU, sensores que registra la orientación, inclinación y aceleración; minimizando así el tamaño del sistema y pudiendo instalarlos en una plataforma portátil.

Una vez analizados todos los diferentes sistemas de medición de viento, se ha utilizado una IMU, en este caso formada por el sensor MPU9250, que contiene a su vez el sensor MPU6050 y el AK8963, el sensor MPU6050 se encarga de la inclinación y orientación mientras que el AK8963 solo de la orientación. Se ha diseñado y codificado un software en python3 para la obtención y gestión de datos, así como para su representación de datos de forma adecuada en las gráficas del capítulo \ref{CAP:RESEXPYDISC}. Se ha instalado el sensor de viento descrito en un cuadrotor, dron de 4 motores, permitiendo así la toma de datos de forma sencilla en diferentes puntos. Para poder instalar la sensorización en el dron se ha tenido que diseñar e imprimir una serie de piezas 3D, ilustradas en la figura \ref{FIG:ACOPLDRONRASPI} (b), permitiendo de esta forma instalar tanto el sensor de viento diseñado como otros posibles tipos de sensorización, como sistemas de medición de temperatura, humedad, presión en incluso una estación meteorológica. Los datos obtenidos mediante el sensor de viento creado son bastante satisfactorios, se puede observar la desviación que produce en viento en el cuadrotor mediante las figuras: \ref{FIG:3axisforces}, \ref{FIG:scatterFit}, \ref{FIG:MAPACALOR}(a),\ref{FIG:MAPACALOR}(b), \ref{FIG:PLANOSMEDIDAS}(a), \ref{FIG:PLANOSMEDIDAS}(b) y \ref{FIG:mapa3d}.

Para proseguir con este trabajo se puede partir por desarrollar un sistema de viento más completo utilizando sensores más sofisticados o precisos como un anemóetro láser doppler o alguno de los descritos en el capítulo \ref{CAP:ESTADODELARTE}. Por otra parte, la plataforma donde se ha instalado el sensor de viento desarrollado, un cuadrotor, se puede modificar instalando otros componentes o se puede llevar a cabo una reestructuración utilizando un dron de ala fija para maximizar de esta forma el tiempo de vuelo pero sacrificando la versatilidad que ofrece un dron de ala rotatoria. Otro campo que puede implementarse es el desarrollo de un sistema de conducción autónomo agilizando la toma de datos.