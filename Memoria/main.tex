\documentclass[spanish]{book}
%poner spanish para indentar todos los parrafos
\usepackage[utf8]{inputenc}
\usepackage[spanish]{babel}
%\usepackage{loram}

\title{CursoLateX}
\author{Daniel Serena Sanz}
\date{April 2019}

\begin{document}

\maketitle

\part{Introducción}
    \chapter{Introduccion}
        \section{Mas Introduction}
            \subsection{subseccion}
                \subsubsection{subsubseccion}
                    %\subsubsubsection{subsubsubsection}
                        \paragraph{parrafo}
                            \subparagraph{subparagraph}
    \textbf{lo que queremos por en negro} poner en negrita
    
    \textit{en italica} texto en italica
    
    \textsl{inclinado} inclinado
    
    \texttt{textttt} textttt
    
     \textsc{mayus} culass
     
    %Otro parrafo \large{grande} o \Large{Muy grande} o \huge {gigante}
    
    %Otro parrafo \small{pequeña} o \footnotesize {muy pequeña} o \scriptsize{mazo pequeño} o \tiny {canija}
    
    Escapar caracteres especiales con BackSlash y para escajar BacksSlash poner \textbackslash
    
\part{Estado del arte}
    \chapter{Estado del arte}
    \section{Sistemas de medicion de viento actuales}
    \section{Sistemas de drones actuales}
    \section{Porque propongo yo esta opcion}
        \subsection{Componentes utilizados}
        \subsubsection{Sensores}
        \subsubsection{Dron}
        \subsubsection{RaspBerry Pi 3B +}

\part{Teoría}
    \chapter{Teoría}

\part{Sistema}

    \chapter{Sistema}
        \section{Sensor de viento}
            \subsection{Elementos}
            Actualmente los sensores de vientos habituales cuentan con dos elementos básicos. Uno de ellos se encarga de medir la velocidad del viento y el otro la dirección de este, a distinguir el anemómetro y la veleta respectivamente. Estos dos elementos pueden encontrarse montados de forma separada o conjuntamente denominandose veleta potenciométrica. 
            
            
        
        \section{Dron}
            \subsection{Construccion dron}
            Para la construcción del dron sobre el que se han realizado las pruebas, se ha seguido una lista de componentes, partiendo de los mas restrictivos y completando con aquellos mas flexibles dependiendo del sistema montado.
                \subsubsection{Motores}
                Los motores son una parte fundamental de un dron ya que han de proporcionar la potencia suficiente para hacer girar la hélices y por consecuente hacer volar al dron. 
                
                He tenido en cuenta varios parámetros como por ejemplo, la potencia, el consumo y la fuerza maxima de empuje. Los motores vienen clasificados segun su velocidad indicandose como KV, revoluciones por minuto por voltio. A su vez se pueden diferenciar entre dos clases de motores, con escobillas o sin ellas, este factor afecta a la forma de cambio de giro de los motores. He elegido unos motores sin escobillas por su rendimiento y menor desgaste que los motores con escobillas. 
                
                Los motores elegidos constan de TODOKV y sin escobillas, contando asi con 3 cables: alimentacion, masa y potencia. Se varía la velocidad de los motores mediante la utilizacion de una señal PWM, variandose desde un 0\% a un 100\% de potencia modificando su ciclo de trabajo desde TODO s a TODO s.
                \subsubsection{ESC}
                El ESC, Electronic Speed Contoller, es un controlador esencial que determina la potencia suministrada al motor, es necesario uno por cada motor. Puede contar con 2 cables o 3 dependiendo si contiene BEC o no, 
                \subsubsection{Hélices}
                \subsubsection{Placa distribuidora de potencia}
                \subsubsection{Controladora de vuelo}
                La placa controladora de vuelo es la unidad de procesamiento del dron, es la encargada de gestionar la señales provenientes del receptor, leer los parámetros de los diferentes sistemas de estabilización y su posterior modificacion de la potencia de cada motor.
                \paragraph{parrafo0 sobre la placa comercial}
                He realizado un estudio sobre las diferentes controladoras de vuelo, comprobando que aceptan las señales del receptor comprado, asi como las posibles entradas y salidas para el numero de motores del dron, sensores extras que se quisiesen instalar.
                \paragraph{parrafo1 sobre la placa hecha por una raspberry}
                En este TFG he intentado realizar una modificacion del sistema de un dron, unificandolo y gestionandolo en un ordenador de 
                \subsubsection{Chasis}
                \subsubsection{Sistema de comunicación}
        
        \section{Sensor de viento en DRON}

\part{Resultados experimentales y discusión}
    \chapter{Resultados experimentales y discusión}

\part{Conclusión}
    \chapter{Conclusión}

\appendix

\chapter{apendice}
\end{document}