En este experimento se pretende evaluar como afecta una inicialización optimista o pesimista de los algoritmos. Para ello fijamos como parámetos de entrada un número de aciertos, que llamaremos $\alpha$ y un número de fallos denotado por $\beta$. En el caso de $\varepsilon$-Greedy, inicializamos el valor de cada brazo según \ref{EQ:VALUE} para todos los items en el conjunto de datos.

\begin{equation}[EQ:VALUE]{Estimación del valor de una acción}
    \ Q(i) = \dfrac{\alpha}{\alpha + \beta}
\end{equation}

El experimento consiste en variar los parámetros $\alpha$ y $\beta$ tomando valores enteros en el intervalo $[0,1]$ y observar el valor del $recall$ a la mitad de las iteraciones. Se obtienen así un total de 100 puntos que se representan el la figura \ref{FIG:EpsOpCm100k} mediante un mapa de color. 

Se observa como los mejores resultados se encuentran para un valor de $\beta$ superior a 6 y para valores de $\alpha$ inferiores a 6, alcanzándose el máximo para $\alpha=3, \ \beta=9$, aunque se encuentran otras zonas con alto recall fuera de esta región, como por ejemplo en $\alpha=10, \ \beta=10$, donde aparece un máximo local.

Se observa asimismo, como para valores de $\alpha$ comprendidos entre 5 y 8, existe una región con valores mínimos para cualquier valor de $\beta$, alcanzándose el mínimo en el punto $\alpha=8, \beta=9$. Aunque de nuevo se observan mínimos locales fuera de esta región ($\alpha=3, \  \beta=4$)

Por tanto se concluye que las esretegias pesimistas resultan positivas para el conjunto de datos de CrowdFlower con el algoritmo de $\varepsilon$-Greedy.

\begin{figure}[Mapa de color de $\varepsilon$-Greedy]{FIG:EpsOpCm100k}{Resultados de la exploración optimista/pesimista}
        \image{7cm}{}{cm100k/MapaEps}
\end{figure}

