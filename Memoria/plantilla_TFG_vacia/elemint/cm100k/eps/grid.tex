\begin{figure}[Resultados de la búsqueda en rejilla para $\varepsilon$-Greedy]{FIG:EPSCM100K}{Resultados para la búsqueda en rejilla}
  \subfigure[sbfig:recalleps]{Comparación del recall para distintos valores de $\varepsilon$}{\image{7cm}{}{cm100k/RecallEps}} \quad
  \subfigure[sbfig:recallmitad]{Corte a mitad del número total de épcas de \ref{sbfig:recalleps}}{\image{7cm}{}{cm100k/RecallMitadEps}}
\end{figure}



La figura \ref{sbfig:recalleps} muestra los resultados tras ejecutar el algoritmo de $\varepsilon$-Greedy para valores de $\varepsilon = 0,\  0.1,\ 0.2,\ 0.4,\ 0.6,\ 0.8,\ 1$. Un valor de $\varepsilon = 1$ equivale a una recomendación aleatoria, mientras que $\varepsilon = 0$ equivale a recomendar el item más popular. Para facilitar la visualización de los resultados, en la figura \ref{sbfig:recallmitad} se proporciona un corte a la mitad de iteraciones (aproximadamente 500) en el que en el eje x se representan los valores de $\varepsilon$ escogidos.

Se observa como el parámetro óptimo es $\varepsilon = 0.1$ y como a partir de ese valor el $recall$ decrece. Es destacable como para valores de $\varepsilon = 0.1,\ 0.2,\  0.4$ se alcanza un $recall$ superior al correspondiente para $\varepsilon = 0$, lo que indica que la exploración es positiva en el experimento.

%%

%\begin{figure}[ejemplo de uso de figure]{fig:subfiguras}{figura de ejemplo. el pie de figura debe ser suficientemente explicativo y con el tamaño que haga falta mientras que el de las subfiguras debe reducirse al mínimo y hacer referencia a las figuras en este pie de figura, como por ejemplo haciendo referencia a la figura \ref{sbfig:dos}}
%  \subfigure[sbfig:una]{esta es una subfigura}{\image{3cm}{}{logoeps}} \quad
%  \subfigure[sbfig:dos]{esta es otra subfigura}{\image{3cm}{}{escudouam}}
%\end{figure}
%%