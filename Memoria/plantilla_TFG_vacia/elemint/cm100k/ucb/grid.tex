\begin{figure}[Resultados de la búsqueda en rejilla para UCB]{FIG:UCBCM100K}{Resultados para la búsqueda en rejilla}
  \subfigure[sbfig:recallucb]{Comparación del recall para distintos valores de $\gamma$}{\image{7cm}{}{cm100k/RecallUCB}} \quad
  \subfigure[sbfig:recallmitaducb]{Corte a mitad del número total de épcas de \ref{sbfig:recallucb}}{\image{7cm}{}{cm100k/RecallMitadUCB}}
\end{figure}

La figura \ref{sbfig:recallucb} muestra los resultados tras ejecutar el algoritmo de UCB para valores del parámetro $\gamma = 0, \ 0.1 \ 1, \ 2, \ 10, \ 100$ en la fórmula \ref{eq:ucb}. %insertar la formula en la descripcion del algoritmo
\begin{equation}[eq:ucb]{UCB}
    I_{t} = \underset{i}{\operatorname{argmax}}\Bigg[ \  Q_{t}(i)+\sqrt{\dfrac{\gamma \ ln(t)}{N_{t}(i)}} \ \Bigg]
\end{equation}
Donde $I_{t}$ denota el item escogido por el algoritmo en cada época, $Q_{t}(i)$ el reward promedio asociado al item $i$ en cada época y $N_{t}(i)$ el número de veces que el item $i$ ha sido seleccionado por el algoritmo.

Cuanto mayor sea el parámetro $\gamma$, mayor peso se le otorga a la incertidumbre, por tanto, el algoritmo tenderá a recomendar los items menos populares en cada iteración. Este comportamiento se puede observar en las figura \ref{sbfig:recallucb}, donde, para valores de $\gamma$ elevados, las curvas se aproximan a la curva de recomendación aleatoria. 
De nuevo, como ocurría en el caso de $\varepsilon$-Greedy, la recomendación basada exclusivamente en la popularidad ($\gamma = 0$) obtiene un peor resultado que dándole cierto peso a la incertidumbre. 

Se concluye, por tanto, que para este conjunto de datos, el parámetro que mejor resultado proporciona es $\gamma = 0.1$.