Los datos de CrowdFlower \cite{cm100k} contienen 103.584 puntuaciones anónimas sobre 1.054 canciones. Para simplificar el exprimento, las puntuaciones se han binarizado, de manera que los items puntuados con 1 o 2 se consideran como 0 (al usuario no le gustó la canción) y los puntuados con 3 y 4 pasan a valer 1. Adicionalmente se dispone de un segundo dato que indica si el usuario ya conocía la canción antes de puntuarla.

A diferencia de un algoritmo de aprendizaje automático convencional, para estos experimentos no se cuenta con un conjunto de entrenamiento, sino que al inicio del experimento todos los datos forman parte del conjunto de test. Conforme se van realizando las sucesivas recomendaciones, se van añadiendo al conjunto de entrenamiento con la información obtenida de test y se actualizan los valores de los items necesarios para continuar con la ejecución del algoritmo.
%.bib para la bibliografia