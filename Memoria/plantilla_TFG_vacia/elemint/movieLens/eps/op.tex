Para mostrar los resultados de la exploración optimista y pesimista, se ha optado nuevamente por un mapa de color, que puede verse en la figura \ref{FIG:EpsOpML}.

\begin{figure}[Mapa de color de Thompson Sampling]{FIG:EpsOpML}{Resultados de la exploración optimista/pesimista}
        \image{7cm}{}{ML/MapaEps}
\end{figure}

La primera diferencia que se observa respecto a los experimentos anteriores es que el $recall$ alcanza valores mucho más elevados, por encima de $0.72$. Asimismo, la zona azul, correspondiente a valores de $recall$ más altos, es bastante más extensa que en resultados anteriores, donde se observaba un mayor equilibrio. De nuevo hay presencia de mínimos locales dentro de la zona azul, pero esta vez no se observan máximos locales en la zona roja.

El mínimo se encuentra para $\alpha = 7, \ \beta = 1$, mientras que se observan varios puntos máximos con valor de $recall$ similar ($\alpha = 1, \ \beta = 5, \ \alpha = 2, \ \beta = 9 y \ \alpha = 5, \ \beta = 8$.

A la vista de \ref{FIG:EpsOpML} se concluye que las estrategias pesimistas conducen a un mejor resultado.