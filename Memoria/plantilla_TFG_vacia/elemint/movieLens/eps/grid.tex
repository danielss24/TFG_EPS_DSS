Los resultados correspondientes a la búsqueda en rejilla se muestran en la figura \ref{FIG:EPSML}.

\begin{figure}[Resultados de la búsqueda en rejilla para $\varepsilon$-Greedy]{FIG:EPSML}{Resultados para la búsqueda en rejilla}
  \subfigure[sbfig:recallepsML]{Comparación del recall para distintos valores de $\varepsilon$}{\image{7cm}{}{ML/RecallEps}} \quad
  \subfigure[sbfig:recallmitadML]{Corte a mitad del número total de épcas de \ref{sbfig:recallepsML}}{\image{7cm}{}{ML/RecallMitadEps}}
\end{figure}

En la figura \ref{sbfig:recallepsML} se observa como las curvas correspondientes a valores de $\varepsilon = 0, \ 0.01, \ y \ 0.001$ son muy similares. En el gráfico \ref{sbfig:recallmitadML} los puntos correspondientes a esos valores se solapan, por ello, en la figura \ref{FIG:ZoomEpsML} se muestra el detalle de \ref{sbfig:recallmitadML} para dichos valores de $\varepsilon$, donde se obtiene que en el caso de MovieLens con el algoritmo de $\varepsilon$-Greedy, la estrategia de popularidad supera a cualquier estrategia de bandidos.

\begin{figure}[Detalle de la exploración para $\varepsilon=0, \ 0.001, \ 0.01$]{FIG:ZoomEpsML}{Detalle de \ref{sbfig:recallmitadML}}
        \image{7cm}{}{ML/RecallMitadEpsbest}
\end{figure}