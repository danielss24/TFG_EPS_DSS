
Un punto crucial de los componentes electrónico es su alimentación, para favorecer la durabilidad y estabilidad del circuito o circuitos electrónico debemos aislar el circuito de potencia del circuito lógico mediantes elementos con la placa distribuidora de potencia (PDB) que cuenta con elementos como diodos, resistencias y transistores para evitar picos de tensión y derivaciones.

Esta placa se encarga de conectarse con la batería mediante un conector, en este caso, XT60; este conector impide conectar cada polo con su opuesto debido a su forma evitando cortocircuitos y daños al sistema.

Por otro lado, la PDB distribuye la tensión necesaria al resto de los componentes, entre los que destacamos:

\begin{description}
        \item[ESC:] Hasta un total de 6 controladores de velocidad, en mi caso son 4 ESC que funcionan con baterías de 2 a 4 celdas y por tanto desde 6 voltios a 16 voltios.
        \item [Circuito 5V:] Circuitos de 5 voltios de tensión con un sistema de BEC para evitar picos y conseguir un circuito con alimentación continua y estable, con una potencia continua de 2 amperios y una potencia máxima de 2'5 amperios.
        \item [Circuito 12V:] Circuitos de 12 voltios de tensión, con el mismo funcionamiento que los circuitos a 5 voltios, con una potencia continua de 500 mA y máxima de 0'8 amperios.
\end{description}

