
 El ESC, Electronic Speed Contoller, es un controlador esencial que determina la potencia suministrada al motor, es necesario instalar uno por cada motor controlandolos asi de forma independiente. 
 Puede contar con 2 cables o 3 dependiendo si es un controlador de velocidad para un motor con escobillas o sin ellas.
 
 \begin{figure}{FIG:ESCDRON}{Controladores de velocidad dron.}
	\image{4cm}{}{escDron}
\end{figure}
 
 Los controladores de velocidad comprados cuentan con una capacidad de pico de descarga máxima de 35 amperios, la cantidad de descarga máxima limita la potencia máxima que pueden generar los motores. Los ESC se deben elegir en función de la batería que vayas a utilizar, teniendo en cuenta el número de celdas y su ratio de descarga continua.
 
 Los ESC pueden recibir la señal que indique la potencia que debe de suministrar al motor, a partir del controlador de vuelo o directamente desde el receptor del mando. Es aconsejable recibirlo desde el controlador de vuelo ya que de esta forma gestiona un microchip la estabilización del dron, favoreciendo a un control mucho mas fácil del aeromodelo a control remoto.\cite{OscarSerrano}

%http://fpvmax.com/2016/12/21/variador-electronico-esc-funciona/
 