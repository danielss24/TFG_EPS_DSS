
 %http://fpvmax.com/2017/02/10/helices-drones-tipos-tamanos/
 Las hélices son un punto menos crítico a la hora de elegirlas y montarlas sobre el dron, contamos con diferentes formatos, formas, número de aspas...
 Podemos encontrar hélices con las siguientes características:
 \begin{description}
        \item[Tamaño:] Puede variar desde 2 a 19 pulgadas, el tamaño de la hélices se ve limitado por el tamaño de los brazos del chasis donde se monte. Hay que tener en cuenta que a mayor tamaño mas superficie y por tanto mas empuje, hoy en día se utilizan mayormente las hélices entre 4 y 6 pulgadas, estas hélices favorecen el funcionamiento de drones con motores rápidos como los utilizados en drones de carreras, no obstante también se utilizan las hélices con tamaño de 10 o mas pulgadas para aeromodelos que no priorizan la velocidad y necesiten estabilidad y llevar cargas elevadas.
 		\item[Número de aspas:] Varía entre 2 a 6 aspas, el número de aspas afecta a la superficie de empuje, aumentando la fricción y por tanto la fuerza que ejerce para levantar el drone. En hélices de tamaño pequeño, hasta 6 pulgadas, podemos encontrar hasta un total de 6 aspas mientras que en hélices de tamaño grande, 10 pulgadas o más, no solemos superar las 2 aspas.
 		\item[Forma:] La forma es un punto crítico de este elemento, tienen una forma de pala corvada para conseguir el empuje del aire en el sentido contrario de giro del motor. La terminación del aspa también afecta a la hora de volar un dron, se pueden diferenciar acabadas en punta,  Bullnose e Híbridas bullnose, determinando mayor o menor empuje y por consecuente mayor o menor consumo.
 Podemos contar con hélices de 2 a 6 aspas, con un tamaño variable desde 2  hasta 19 pulgadas.
 
 \end{description}
 
 En el dron diseñado he priorizado el tamaño, como he explicado anteriormente el tamaño de las hélices afecta a la fuerza de empuje vertical del dron para poder levantar su carga, no obstante dado que contamos con un dron con un peso máximo de NOSECUANTOS gramos y con una carga añadida de 200g máx., he elegido unas hélices de 5 pulgadas con un total de 3 aspas, obtenidendo asi un punto óptimo entre empuje o fuerza máxima y duración de vuelo.
 

 