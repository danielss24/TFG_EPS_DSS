
Los motores son una parte fundamental de un dron ya que han de proporcionar la potencia suficiente para hacer girar la hélices y por consecuente hacer volar al dron. 
                
                He tenido en cuenta varios parámetros como por ejemplo, la potencia, el consumo y la fuerza maxima de empuje. Los motores vienen clasificados según su velocidad indicándose como KV, revoluciones por minuto por voltio. A su vez se pueden diferenciar entre dos clases de motores, con escobillas o sin ellas, este factor afecta a la forma de cambio de giro de los motores. He elegido unos motores sin escobillas por su rendimiento y menor desgaste que los motores con escobillas. 
                
                Los motores elegidos constan de TODOKV y sin escobillas, contando así con 3 cables: alimentación, masa y potencia. Se varía la velocidad de los motores mediante la utilización de una señal PWM, variándose desde un 0\% a un 100\% de potencia modificando su ciclo de trabajo desde TODO s a TODO s.