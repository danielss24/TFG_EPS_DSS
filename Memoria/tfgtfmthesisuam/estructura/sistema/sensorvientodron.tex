
En esta sección describiré como uno el sensor de viento con el dron y que es necesario para que todo funcione como un único sistema.
Para poder llevar a cabo esta unión se ha de tener en cuenta el peso del sistema de viento, raspberry y sensorización, junto con el peso del dron para hacer una estimación media de la duración de vuelo.


%\paragraph{Uniendo sistemas}
\begin{description}
\item[Desde donde parto]
Para poder diseñar un sensor de viento que funcione con un dron primero hay que estudiar que sistema de estabilización usa el dron, podemos diferenciar entre dos principales sistemas. Un primer sistema realiza una estabilización del dron que registra la desviación de su posición desde un punto y lo corrige para mantenerse siempre en el mismo punto. El segundo sistema de estabilización intenta que el dron no se caiga modificando la potencia de los motores acorde con la fuerza que se ejerza sobre el, de esta forma consigue que el dron esté los más horizontal posible.

En mi caso, uso un sistema comercial que funciona como el segundo sistema descrito y por tanto uso un giroscopio para poder medir dichas variaciones. En el caso de usar el primer sistema se debería usar una medición de la potencia suministrada a los motores para ver que potencia esta ejerciendo para contrarrestar la fuerza que se ejerce sobre él.

  \begin{figure}{FIG:FOTOWARNING}{PONER FOTO DE DRON VIRGEN Y RASPY/SENSOR.}
	\image{4cm}{}{warningFoto}
\end{figure}

Por otro lado contamos con un chasis base de un dron de 220 mm de longitud de hélice a hélice donde es necesario montar un soporte para instalar nuestro ordenador o unidad de procesamiento, en mi caso una Raspberry Pi 3 B+, y nuestro sistema de sensorización.

\item[Como lo junto] Para poder unir los dos sistemas y realizar medidas en vuelo, he tenido que diseñar una serie de piezas 3D para sus posterior impresión y acoplamiento en el chasis del dron.

Entre las piezas de impresión podemos contar con:
\begin{description}
\item[Soporte Raspberry Pi]
Para el soporte de la Raspberry Pi 3 B+, he diseñado una pieza que se acopla en la parte inferior del chasis del dron. 

\begin{figure}[Chasis raspi dron]{FIG:CHASISRASPIDRON}{Soporte Raspberry dron.}
	\image{4cm}{}{ChasisRaspberryDron}
\end{figure}

\item[Zona de sensorización: ]

Una vez instalado el soporte de la Raspberry he diseñado una tapa intermedia donde se puede instalar el sistema de sensorización necesario. Esta zona esta diseñada con suficiente espacio para instalar mas sensores a parte del sensor base diseñado, sensor de viento, como por ejemplo un sensor de gas con sus componentes necesarios.

\begin{figure}[Zona sensorización dron]{FIG:ZONASENSORIZACION}{Zona de sensorización. Tapa intermedia \ref{SBFIG:TAPAINTERMEDIA} acoplada al soporte de la raspberry \ref{FIG:CHASISRASPIDRON} y a ella se ancla la tapa \ref{SBFIG:TAPACHASISDRON} para cerrar nuestro modelo.}
	
  \subfigure[SBFIG:TAPAINTERMEDIA]{Tapa intermedia dron}{\image{4cm}{}{TapaIntermediaDron}} \quad
  \subfigure[SBFIG:TAPACHASISDRON]{Tapa chasis dron}{\image{4cm}{}{TapaChasisDron2}}
  
\end{figure} 

Al diseñarlo de forma modular, por piezas, podemos modificar la tapa del chasis del dron \ref{SBFIG:TAPACHASISDRON} y diseñar otra forma o ampliar el espacio para que se acople a la tapa intermedia  \ref{SBFIG:TAPAINTERMEDIA}.

\item[Elementos de aterrizaje]

Una vez instalado las anteriores piezas, hemos prolongado el dron por su parte inferior un total de TANTOS cm y es necesario instalar un tren de aterrizaje.

\begin{figure}[Pata dron]{FIG:PATADRON}{Diseño de pata del dron del tren de aterrizaje.}
	\image{4cm}{}{PataDron}
\end{figure}
\end{description}

Para el tren de aterrizaje se ha probado instalar un total de 4 patas, una por cada motor. Por problemas de estabilidad se eliminó una de ellas y la tercera se instaló en la parte delantera-media del dron. Con 3 patas nos aseguramos aterrizar el dron en una superficie plana.

\item[Como funciona todo junto]

Una vez unido todo como un solo sistema, dron y sensorización junto con la raspberry, procederé a realizar las mediciones con el dron volando.

Para tomar las medidas correctamente el dron ha de ponerse en un sitio y mantenerlo volando. Se procede a la toma de medidas y con la variación de su posición en cuanto a inclinación que sufre por fuerzas como el viendo obtenemos los datos.
Para la realización de esta prueba contamos con los cálculos de vuelo previamente hechos sobre un dron con un peso máximo de 900 gramos y con una batería de 3 celdas y 1500 mAh de capacidad. El unico manejo que se ha de hacer con el dron a la hora de medir es modificar su altura ya que no tiene un sistema automático de mantenimiento de distancia respecto del suelo.

  \begin{figure}{FIG:FOTOWARNING}{FOTO DEL SISTEMA FINAL.}
	\image{4cm}{}{warningFoto}
\end{figure}


\end{description}

