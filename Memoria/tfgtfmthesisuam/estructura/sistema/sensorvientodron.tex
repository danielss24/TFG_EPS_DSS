
En esta sección describiré como junto el sensor de viento con el dron y que es necesario para que todo funcione como un único sistema.
Para poder llevar a cabo esta unión se ha de tener en cuenta el peso del sistema de viento, raspberry y sensorización, junto con el peso del dron para hacer una estimación media de la duración de vuelo.


%\paragraph{Uniendo sistemas}
\begin{description}
\item[Desde donde parto]

\item[Como lo junto] Para poder juntar los dos sistemas y realizar medidas en vuelo, he tenido que diseñar una serie de piezas 3D para sus posterior impresión y acoplamiento en el chasis del dron.
Entre las piezas de impresión podemos contar con:
\begin{description}
\item[Acoplamiento Raspberry Pi]
Acoplamiento Raspberry Pi 3 B+

\begin{figure}{FIG:CHASISRASPIDRON}{Chasis raspberry dron.}
	\image{4cm}{}{ChasisRaspberryDron}
\end{figure}

\item[Zona de sensorización: ] Zona de sensorización del dron.


\begin{figure}[Zona sensorización dron]{FIG:ZONASENSORIZACION}{Zona de sensorización}
	
  \subfigure[SBFIG:TAPAINTERMEDIA]{Tapa intermedia dron}{\image{4cm}{}{TapaIntermediaDron}} \quad
  \subfigure[SBFIG:TAPACHASISDRON]{Tapa chasis dron}{\image{4cm}{}{TapaChasisDron2}}
  
\end{figure} 

\item[Elementos de aterrizaje]
Elementos de aterrizaje

\begin{figure}{FIG:PATADRON}{Avión con los ejes de giro, yaw, pitch y roll, obtenidos mediante cuaterniones.}
	\image{4cm}{}{PataDron}
\end{figure}
\end{description}

\item[Como funciona todo junto] Funciona mu bien mu bien.
\item[Como lo utilizo] Con las manos por lo general.
\item[Que he conseguido] Un sensor guay guay.
\end{description}

