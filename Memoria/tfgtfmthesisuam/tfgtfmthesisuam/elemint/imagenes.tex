  Como elemento previo es interesante recordar la resolución mínima necesaria de una imagen para ser impresa correctamente. Una resolución mínima implica 120 puntos por centímetro (o los que es lo mismo 300 ppp). Es decir, si vamos a presentar una imagen con pixel cuadrado con un ancho de 12 cm en el papel deberá tener como mínimo 120 $\times$ 12 = 1440 puntos. Cualquier imágenes con una resolución inferior no se verá con la nitidez adecuada. Por otro lado, para una resolución de impresión con calidad fotográfica de 1200 ppp, o lo que es lo mismo 480 puntos por centímetro la resolución necesaria para el caso anterior sería 480 $\times$ 12 = 5760 puntos, sin embargo resoluciones tan altas si se tienen muchas imágenes se puede relentizar mucho la compilación del documento.

 	Para introducir una imagen se utiliza la función \textbf{{\textbackslash}image} que tiene tres parámetros obligatorios de los cuales los dos primeros se pueden dejar vacíos. el primer parámetro es el ancho de la imagen; el segundo es la altura de la imagen y el tercero es el nombre de la imagen sin extensión. Se admiten múltiples formatos de ficheros entre los que se incluyen JPG, GIF y PNG. Si se deja vacío el ancho o el alto, la dimensión no definida se calcula internamente para mantener la relación de aspecto original. Si se dejan los dos vacíos la imagen utiliza el 90\% del ancho del texto y la altura se calcula para mantener la relación de aspecto. Un del uso de este comando puede verse en el código de las figuras \ref{FIG:EJEMPLO} y \ref{FIG:SUBFIGURAS}.

  Además se dispone de un comando para introducir imágenes \textit{in line} que van a ser rodeadas por el texto. Este comando es el \textbf{{\textbackslash}imageIL} que tiene dos parámetros, el primero es el nombre de la imagen y el segundo es su anchura en cualquier unidad permitida por \LaTeXe.
