
La realización del estudio del viento con tecnologías mas modernas se puede calificar de superficial por la gran variedad de sensores que existen hoy en día. Utilizando como sensor principal el sensor MPU9250, ha proporcionado unas medidas bastante exactas en cuanto a dirección y evolución del viento, no obstante habría que comparar la fuerza del viento medida con un sistema ya probado y afianzado como un anemómetro descrito en la sección \ref{SEC:SENSORVIENTO}. 

Como se puede observar en las gráficas expuestas en \ref{SEC:EXPOSICION} hay una variación acorde con el foco de viento generado, lo cual corrobora el trabajo realizado.

Para continuar con este estudio y obtener mejores resultados, se debería probar un dron con ambos sistemas de estabilización y obtener los datos tanto de un sensor como el utilizado como de la potencia suministrada a los motores para mantenerse sobre un mismo punto. 