
He realizado una toma de medidas de formas distintas: la primera como una toma única de datos y la segundo como la toma de conjunto de puntos como el descrito en la figura \ref{FIG:planoGimnasio}.

En la toma de medidas de ambas formas he alterado la altura del dron para poder realizar una comparación del viento, a una altura de 1 metro (m) del suelo y a 2,5 o más metros. Cabe indicar que para toma del conjunto de datos, he realizado simulado una matriz en el gimnasio y he ido modificando la posicion de dron para cada una de las tomas de datos, cambiando tanto su posicion en el eje X, Y y Z.

Para la toma de medidas he simulado una corriente de aire con un ventilador, no obstante al ser una circulación de aire tan leve incorporé en la generación del aire, un movimiento mecánico manual de una colchoneta a forma de abanico.

Una vez preparado el escenario, con y sin aire, procedí a la colocación del dron y a la toma de datos. En una primera prueba realice tomas de datos individuales a varias alturas para familiarizarme con el comportamiento del dron con y sin viento. En las siguientes pruebas realicé, algunas tomas individuales y un total de 2 pruebas de matrices, con un total de 8 puntos, 4 en un plano y 4 en otro plano. El primer plano lo definí a una altura aproximada de 0'30 metros (m) del suelo, mientras que el segundo lo situé a una altura aproximada de 2-2'5 m de altura. 

La forma de tomar los datos con el sensor de viento y el dron, fue iniciada y terminada en vuelo.