
Los datos obtenidos mediante esa serie de pruebas son los siguientes.

\begin{figure}[Aceleraciones en 3 ejes.]{FIG:3axisforces}{Aceleración en fuerza G que sufren los 3 ejes cartesianos. En esta gráfica podemos observar 3 líneas rojas, una por cada plot. Estas línas indican el estado de reposo del acelerómetro. Entorno a ellas podemos ver como varía la fuerza ejercido sobre cada eje, X,Y y Z en las gráficas a,b y c, respectivamente.}
	\image{1\textwidth}{}{3axisforces}
\end{figure}

\begin{figure}[Puntos de medida]{FIG:scatterFit}{Puntos tomados en funcion de su orientación y en fuerza G. La recta roja representa la representación de los datos, hacia que dirección van.}
	\image{13cm}{}{scatterFit}
\end{figure}

\begin{figure}[Mapa de calor con zoom]{FIG:mapaCalorCZ}{Mapa de calor con representación de concentración de puntos delimitado con isobaras. Podemos observar que tenemos una concentración mayor entorno al (0.0,0.0) y una leve variación hasta un máximo de -0.1 y 0.15 en el eje X y -0.6 y + 0.6 en el eje Y. Esta muestra de datos que no había apenas aire.}
	\image{0.5\textwidth}{}{mapaCalor2}
\end{figure}

\begin{figure}[Mapa de calor sin zoom]{FIG:mapaCalorSZ}{Mapa de calor con escala máxima de 4 G tanto positivos como negativos. Podemos observar que no hay casi variación y tenemos la concentración de los puntos esta entorno al origen de la gráfica.}
	\image{0.5\textwidth}{}{mapaCalorSZ2}
\end{figure}

\begin{figure}[Mapa viento 2D - Z0]{FIG:z0MapaViento}{Podemos observar una serie de vectores obtenidos mediante la toma de datos en forma de matriz en el gimnasio de IES Valle Inclán. Se puede apreciar la magnitud y dirección del aire, aunque no su tercera componente, altura que toma el vector.}
	\image{0.5\textwidth}{}{z0MapaViento}
\end{figure}


\begin{figure}[Mapa viento 2D - Z1]{FIG:z1MapaViento}{Observamos ahora el plano superior de la toma de datos en forma de matriz. Se muestra dirección y magnitud.}
	\image{0.5\textwidth}{}{z1MapaViento}
\end{figure}

\begin{figure}[Representación toma de puntos en gimnasio]{FIG:gimnasioPtsReales}{Descripcion.}
	\image{1\textwidth}{}{gimnasioPanoramicaPtsToma}
\end{figure}

\begin{figure}[Mapa de viento 3D]{FIG:mapa3d}{En esta gráfica 3D muestro todos los conjuntos de datos tomados en forma de matriz de 2x2x2, observando la dirección, en 3 ejes y la fuerza que hay en cada uno de los puntos. Como se puede observar, el viento provocado desde el punto 0 influye se ve representado en cada una de las medidas en cuando a dirección y fuerza. Aunque en el punto 4 no afecte el viento generado, dado que creamos una corriente de aire, provoque cierta absorción desde la puerta del almacén, forzando asi a crear una corriente de aire demostrada en la gráfica.}
	\image{0.8\textwidth}{}{mapa3d2}
\end{figure}